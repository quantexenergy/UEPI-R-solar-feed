% ========================================================================
% UEPI-R: A Real-Time Early Warning System for Solar Flares
% Using Causal Regime Detection on GOES XRS Data
% ========================================================================
\documentclass[twocolumn,showpacs,preprintnumbers,amsmath,amssymb,aps,floatfix]{revtex4-2}

\usepackage{graphicx}
\usepackage{amsmath}
\usepackage{amssymb}
\usepackage{booktabs}
\usepackage{hyperref}
\usepackage{xcolor}
\usepackage{multirow}
\usepackage{siunitx}

\hypersetup{
  colorlinks=true,
  linkcolor=blue,
  citecolor=blue,
  urlcolor=blue
}

\begin{document}

\title{UEPI-R: A Real-Time Early Warning System for M- and X-Class Solar Flares\\
Using Causal Regime Detection on GOES XRS Data}

\author{Jorge Alexander Castillo}

\date{\today}

\begin{abstract}
We present UEPI-R (Universal Early Pre-Ignition---Regime), a fully causal,
real-time early warning system for M-class ($\geq$M1.0) and X-class solar
flares.  Unlike the majority of existing forecasting systems that rely on
magnetogram-derived features, active region classifications, or deep learning
on multi-instrument imagery, UEPI-R operates exclusively on the GOES X-Ray
Sensor (XRS) 1--8~\AA\ irradiance channel, requiring no spatial solar data
and no supervised training on labeled flare events.  The algorithm performs
causal regime detection on the real-time XRS flux stream, identifying
characteristic statistical signatures that precede major flare onset.

Validated on 16 years of continuous GOES data (2010--2025; 5{,}747 coverage
days, 2{,}638 M/X-class flares, 141 X-class flares), UEPI-R achieves
64--71\% M/X coverage (97.2\% X-class) with $\sim$39\% precision,
a false alert rate of 0.36--0.39~day$^{-1}$, and a day-level True Skill
Statistic of $\approx$0.41, depending on matching methodology.  The median lead time ranges from 2.4~hours (overlap matching)
to 4.3~hours (one-to-one greedy matching), with 80\% of detections
providing $\geq$1~hour of advance warning.  In a high-sensitivity
configuration, coverage reaches 78.2\% (98.6\% X-class).  A live deployment
running continuously since January 2026 against NOAA/SWPC real-time feeds
has produced 6 alerts, 5 of which were verified hits against the official
NOAA flare event list (83.3\% hit rate).  Analysis of alert duration
reveals an inverse correlation with lead time (Spearman $r = -0.29$):
shorter alerts provide longer advance warning, while longer alerts
typically encompass the flare itself.  Crucially, 95.8\% of nominally
``false'' alerts (no M-class follow-up) are associated with C-class
flare activity, with only 4.2\% representing true noise triggers---indicating
that the system detects genuine solar activity in nearly all cases.
These results are competitive with or exceed
operational systems that require far richer input data, demonstrating that
the pre-flare XRS signal alone carries substantial predictive information.
\end{abstract}

\maketitle

% ========================================================================
\section{Introduction}
\label{sec:intro}
% ========================================================================

Solar flares---sudden releases of magnetic energy in the solar corona---pose
significant hazards to technological infrastructure, including satellite
electronics, high-frequency radio communications, and high-altitude radiation
exposure \citep{benz2017, fletcher2011}.  The NOAA Space Weather Prediction
Center (SWPC) issues daily probabilistic forecasts for M- and X-class flares,
which have been extensively validated \citep{crown2012, camporeale2025}. These
operational forecasts, along with most research forecasting systems, rely on
photospheric magnetogram data from instruments such as SDO/HMI, combined with
active-region classification schemes \citep{mcintosh1990, schrijver2007,
bobra2015} and, increasingly, deep learning architectures
\citep{nishizuka2021, abduallah2023, georgoulis2021}.

While magnetogram-based approaches leverage the physical relationship between
photospheric magnetic complexity and flare productivity, they face several
practical limitations: (i)~dependence on a single instrument (SDO/HMI) with
no operational backup, (ii)~reliance on active-region identification and
segmentation pipelines, (iii)~inherent latency in magnetogram processing and
dissemination, and (iv)~forecast horizons typically discretized at
24-hour intervals rather than providing continuous real-time alerts.

An alternative, complementary approach is to exploit the soft X-ray irradiance
signal itself.  The GOES XRS instruments provide continuous, full-disk,
one-minute cadence measurements that are available in near-real-time from
multiple redundant spacecraft.  While the conventional view holds that the XRS
flux primarily records flare \emph{consequences} rather than \emph{precursors},
recent work has demonstrated that XRS time-series data alone can achieve
forecasting skill comparable to magnetogram-based systems:
\citet{riggi2025} report TSS~$\approx 0.74$ for $\geq$M-class prediction
using a foundation time-series model (Moirai2) applied exclusively to
GOES XRS data, outperforming their own magnetogram-based image and video
models.  More broadly, the pre-flare corona often exhibits characteristic
thermal and statistical signatures in the minutes to hours before major
eruptions---reflecting processes such as gradual energy buildup, precursor
heating, and destabilization of the coronal magnetic field.

In this work we present UEPI-R (Universal Early Pre-Ignition---Regime), a
system that performs causal regime detection on the GOES XRS flux stream to
generate real-time early warnings for M- and X-class flares.  UEPI-R requires
no spatial solar data and no ancillary instrument data beyond the two standard
XRS channels (1--8~\AA\ and 0.5--4~\AA).  Its detection criteria are
physics-motivated rather than learned from labeled examples, though a small
number of internal thresholds are optimized on historical data
(Section~\ref{sec:tuning}).  The algorithm is fully causal---each output
depends only on data available at or before the current time step---and can
run on commodity hardware with negligible latency.

The remainder of this paper is organized as follows.
Section~\ref{sec:data} describes the GOES XRS data used for both backtesting
and live operation.  Section~\ref{sec:method} provides a high-level overview
of the detection methodology.  Section~\ref{sec:results} presents results
from the 16-year backtest and the ongoing live deployment.
Section~\ref{sec:comparison} places these results in context with existing
operational and research forecasting systems.
Section~\ref{sec:discussion} discusses limitations, failure modes, and future
directions, and Section~\ref{sec:conclusion} concludes.

% ========================================================================
\section{Data}
\label{sec:data}
% ========================================================================

\subsection{GOES XRS Observations}

UEPI-R uses the GOES X-Ray Sensor (XRS) 1--8~\AA\ (``XRS-B'') irradiance
channel as its primary input, with the 0.5--4~\AA\ (``XRS-A'') channel
providing an auxiliary spectral hardness ratio.  The data span 16 years from
2010 January through 2025 December, encompassing all of Solar Cycle~24 and
the rising phase and maximum of Solar Cycle~25.

For the legacy GOES era (2010--2016), we use the NOAA/NCEI archived
one-minute-averaged XRS data from GOES-13, -14, and -15, distributed as
yearly NetCDF files.  Quality flags are applied, resulting in
approximately 50\% valid data coverage during the legacy period---consistent
with known instrument calibration and eclipse-masking intervals.

For the GOES-R era (2017--2025), we use the operational one-minute XRS-B and
XRS-A products from GOES-16, -17, and -18, obtained from the NOAA/SWPC
archives as daily CSV files.  GOES-R data quality is substantially higher,
with typical coverage exceeding 99\%.

Across both eras, the dataset comprises 5{,}747 effective coverage days
(accounting for data gaps).

\subsection{Flare Catalog}

Ground-truth flare events are obtained from the NOAA/SWPC official event
lists.  Over the 2010--2025 period, the catalog contains 2{,}638 events of
class M1.0 or greater, of which 141 are X-class.  The distribution by solar
cycle phase is highly non-uniform: Solar Cycle~24's maximum
(2011--2015) produced the bulk of events, while the deep minimum of
2018--2019 recorded zero M-class flares.  Solar Cycle~25's maximum
(2023--2025) has been exceptionally active, with 2024 alone contributing
874 M-class and 54 X-class events.

\subsection{Real-Time Data Feed}

For live operation, UEPI-R ingests the NOAA/SWPC real-time JSON feeds
(\texttt{xrays-1-day.json}), which provide the most recent 24~hours of
one-minute XRS-B and XRS-A measurements with a typical latency of
1--2~minutes from observation.  A rolling buffer maintains sufficient
history to support the algorithm's baseline computation.

% ========================================================================
\section{Method}
\label{sec:method}
% ========================================================================

The details of the UEPI-R detection algorithm are proprietary (U.S.\
Provisional Patent Application No.\ 63/949,419, filed December~28, 2025)
and are not disclosed here.  We provide a high-level description
sufficient to characterize the approach and its assumptions.

\subsection{Overview}

UEPI-R performs \emph{causal regime detection} on the real-time GOES XRS
irradiance stream.  The fundamental premise is that the transition from a
quiescent or gradually evolving coronal state to a flare-productive regime
manifests as a detectable shift in the statistical properties of the XRS flux
time series---specifically in properties related to the rate, persistence, and
spectral character of flux enhancements relative to recent background levels.

The algorithm continuously computes a set of dimensionless diagnostic
quantities from the XRS-B flux (and optionally XRS-A for spectral hardness
information), using only data available at or before the current time step
(strict causality).  These diagnostics characterize the current state of the
X-ray corona relative to its recent statistical baseline.  When specific
combinations of these diagnostics satisfy internally defined criteria, the
system transitions into an elevated alert state, indicating that the
statistical regime is consistent with an impending M- or X-class flare.

\subsection{Diagnostic Categories}

While the specific formulations are proprietary, the diagnostics fall into
four conceptual categories that reflect known pre-flare physics:

\begin{enumerate}
  \item \textbf{Baseline deviation.}  The current flux level is compared
    to a causally computed, adaptive baseline representing the recent
    quiescent state.  Sustained elevation above baseline indicates
    increased coronal energy loading.
  \item \textbf{Rate-of-change statistics.}  The temporal derivative of
    the residual flux (after baseline removal) captures the dynamics of
    energy injection.  Persistent positive trends---as opposed to
    transient spikes---are characteristic of the gradual energy buildup
    that precedes major eruptions.
  \item \textbf{Statistical persistence.}  Short-duration flux
    enhancements (e.g., microflares, instrumental artifacts) are
    distinguished from sustained regime shifts through temporal
    persistence criteria.  The algorithm requires that multiple
    diagnostics remain elevated for a minimum duration before committing
    to an alert state.
  \item \textbf{Spectral hardness.}  The ratio of XRS-A (0.5--4~\AA) to
    XRS-B (1--8~\AA) flux provides a temperature-sensitive diagnostic.
    Pre-flare coronal heating produces detectable spectral hardening that
    serves as an independent confirmation of genuine flare precursor
    activity, as opposed to background fluctuations or instrumental drift.
\end{enumerate}

The alert state is managed as a finite state machine with hysteresis:
transitions into elevated states require multiple criteria to be satisfied
simultaneously, while release from an alert state requires sustained
return to baseline conditions, preventing rapid oscillation during
marginal periods.

\subsection{Key Properties}

The following properties distinguish UEPI-R from existing approaches:

\begin{enumerate}
  \item \textbf{XRS-only input.}  The system requires no magnetogram,
    EUV, or radio data---only the two standard GOES XRS channels.
  \item \textbf{No supervised training.}  The algorithm's structure and
    diagnostic categories are derived from physical reasoning about
    pre-flare coronal signatures, not from supervised learning on labeled
    events.  A small number of operating-point thresholds are tuned on
    historical data (Section~\ref{sec:tuning}), but the functional form
    is fixed \emph{a priori}.
  \item \textbf{Strict causality.}  All computations are causal: rolling
    statistics, baselines, and state transitions use only past and present
    data.  No centered or forward-looking windows are employed.
  \item \textbf{Continuous operation.}  Alerts are generated (or released)
    at every one-minute time step, providing sub-minute response to
    changing conditions.
  \item \textbf{Multi-tier alerts.}  The system produces graded alert levels,
    from an initial ``approaching'' state to a full ``red'' alert, allowing
    users to select their preferred sensitivity--specificity tradeoff.
  \item \textbf{Spectral hardness gating.}  The XRS-A/XRS-B ratio provides
    an independent confirmation channel: genuine pre-flare heating produces
    spectral hardening that background fluctuations typically do not.
\end{enumerate}

\subsection{Evaluation Protocol}

We evaluate UEPI-R using two complementary alert-to-flare matching methods,
each capturing a different aspect of operational performance:

\paragraph{Hazard-window matching (1:1 greedy).}
Each alert can match at most one flare (the first M1.0+ event occurring
within a defined hazard window after alert onset), and each flare can be
matched by at most one alert (greedy first-in-time assignment).  This
prevents inflated hit rates during storm periods when multiple flares occur
in rapid succession.  All 2{,}638 catalog events serve as the denominator,
providing the most conservative coverage estimate.

\paragraph{Temporal overlap matching.}
A flare is counted as detected if the alert was active (in an elevated
state) at any point during a window surrounding the flare onset.  This
method better reflects operational utility---whether the system was in an
alert state when the flare occurred---and excludes events during data
gaps from the denominator (1{,}921 evaluable events).

False alerts are counted independently under both methods: any alert whose
entire validity window expires without an M1.0+ flare is counted as false,
regardless of matching.  Coverage days are computed from the actual span of
valid flux data, not from calendar boundaries, ensuring that metrics are
not biased by data gaps.

All backtests use a 48-hour warm-up period at the start of each contiguous
data segment to allow baseline statistics to stabilize.

\subsection{Parameter Tuning}
\label{sec:tuning}

While UEPI-R's diagnostic categories and state-machine architecture are
derived from physical reasoning, the system contains a small number of
operating-point thresholds (on the order of ten parameters) that control
the sensitivity--specificity tradeoff.  These thresholds were tuned via
grid search on the 2010--2025 historical dataset, optimizing the $F_2$
score (which weights coverage twice as heavily as precision) under the
1:1 greedy matching protocol.

To mitigate overfitting, we employed a two-stage protocol: an initial
sweep over 8 representative years spanning both solar cycles (including
years of high, moderate, and minimal activity), followed by validation
on all 16 years.  The stage-1-to-stage-2 performance degradation was
approximately 4\%, indicating that the tuned parameters generalize across
solar cycle phases.

As a further robustness check, we evaluated performance independently on
Solar Cycle~24 (2010--2019; 749 M-class, 49 X-class events) and Solar
Cycle~25 (2020--2025; 1{,}748 M-class, 92 X-class events).  Cycle~24
M-class coverage is 75.8\% and Cycle~25 is 66.8\%, with X-class coverage
of 100\% and 95.7\% respectively.  The M-class gap is driven almost
entirely by 2024, which saw 928 $\geq$M1.0 events---the most active year
in the dataset---where the 1:1 matching constraint becomes binding
(fewer unique alerts than flares during storm clusters).  Excluding 2024,
Cycle~25 M-class coverage is 78.9\%, consistent with Cycle~24.
The eight held-out years not used during stage-1 tuning achieve 80.8\%
M-class coverage, compared with 65.1\% for the eight tuning years---the
opposite of what systematic overfitting would produce.

We emphasize that this tuning adjusts the \emph{operating point} of a
fixed detection architecture---analogous to selecting a decision
threshold on a receiver operating characteristic (ROC) curve---rather
than learning the detection features themselves.  No per-event labels
are used during real-time operation; the algorithm's causal diagnostics
are computed identically regardless of whether a flare subsequently occurs.

% ========================================================================
\section{Results}
\label{sec:results}
% ========================================================================

\subsection{16-Year Backtest (2010--2025)}

Table~\ref{tab:backtest} summarizes the performance of the
precision-optimized UEPI-R configuration under both matching methods,
as well as a high-sensitivity configuration that maximizes coverage.
The precision-optimized configuration targets the best $F_2$ score
(coverage-weighted harmonic mean).

\begin{table*}[t]
\centering
\caption{UEPI-R backtest performance on 2010--2025 GOES XRS data (5{,}747
  coverage days, 141 X-class events).  The precision-optimized configuration
  is evaluated under both matching methods; the high-sensitivity
  configuration uses 1:1 matching.  Lead times are computed over matched
  flares.}
\label{tab:backtest}
\begin{tabular}{llcccccccc}
\toprule
Configuration & Matching & M/X Cov. & X Cov. & Precision & False/day &
  Med.\ Lead & Mean Lead & Hits/$N_\mathrm{events}$ \\
\midrule
Precision-opt.\ & 1:1 greedy
  & 63.9\% & 97.2\% & 39.6\% & 0.36 & 257~min & 375~min & 1{,}685/2{,}638 \\
Precision-opt.\ & Overlap
  & 71.0\% & 97.2\% & 39.1\% & 0.39 & 146~min & 178~min & 1{,}364/1{,}921 \\
High-sensitivity & 1:1 greedy
  & 78.2\% & 98.6\% & 18.5\% & 1.34 & 198~min & 313~min & 2{,}064/2{,}638 \\
\bottomrule
\end{tabular}
\end{table*}

Several features of these results merit emphasis:

\paragraph{X-class coverage.}  All configurations detect $\geq$97\% of
X-class flares (137/141 under precision-optimized, 139/141 under
high-sensitivity).  Only 2--4 X-class events in 16 years are missed,
indicating that the pre-flare XRS signature for the most energetic events
is highly reliable.

\paragraph{Lead times.}  Under overlap matching---which measures the
actual lead time from when the system entered its alert state to flare
onset---the median lead time is 2.4~hours with a mean of 3.0~hours.
Under 1:1 greedy matching, lead times appear longer (median 4.3~hours)
because the earliest alert in a sequence is assigned to the first matching
flare, biasing toward longer leads.  Both perspectives are operationally
informative: the overlap lead times reflect typical operator experience
(how much warning before the flare), while the 1:1 leads reflect earliest
possible detection.

The 10th~percentile lead time under overlap matching is 27~minutes, and
the 90th~percentile is 5.5~hours.  Approximately 80\% of detected flares
are caught with $\geq$1~hour of lead time, 60\% with $\geq$2~hours,
and 92\% within 6~hours (Table~\ref{tab:leadtime}).

\paragraph{Precision--coverage tradeoff.}  Precision is consistent across
both matching methods ($\approx$39--40\%), confirming that the false-alert
rate is robust to evaluation methodology.  The precision-optimized
configuration achieves roughly 2-in-5 alerts followed by an M1.0+
flare---a false alarm ratio of approximately 61\%, compared to $>$90\% for
NOAA/SWPC operational forecasts \citep{camporeale2025}.  The
high-sensitivity configuration increases coverage to 78\% (98.6\% X-class)
at the cost of reduced precision (18.5\%) and a higher false-alert rate
(1.34~day$^{-1}$ vs.\ 0.36~day$^{-1}$).

\paragraph{Skill scores.}  To facilitate comparison with the broader
forecasting literature, we convert UEPI-R's continuous alert output to a
per-day dichotomous format: a day is ``forecast positive'' if any alert was
active during it, and ``observed positive'' if at least one M1.0+ flare
occurred.  Table~\ref{tab:skill} reports the resulting skill scores.  The
True Skill Statistic (TSS~$= \mathrm{POD} - \mathrm{POFD}$) of 0.41 and
Heidke Skill Score (HSS) of 0.35 confirm that the system provides
meaningful skill well above the no-skill baseline, though direct comparison
with 24-hour binary classifiers (which report TSS~$\approx 0.7$--$0.9$
on curated test sets) is not straightforward: UEPI-R is a continuous
onset-warning system evaluated over 16 uninterrupted years including solar
minimum, not a daily binary classifier evaluated on balanced partitions.
The operational gap is instructive: Deep Flare Net reports laboratory
TSS~$\approx 0.80$ but operational TSS~$\approx 0.24$ \citep{kubo2017},
a degradation that UEPI-R's physics-based architecture is designed to avoid.

\begin{table}[h]
\centering
\caption{Day-level skill scores for the precision-optimized configuration
  (2010--2025, 5{,}747 days).  The contingency table is constructed by
  classifying each day as alert-active or quiet and as flare-day or
  non-flare-day.}
\label{tab:skill}
\begin{tabular}{lc}
\toprule
Metric & Value \\
\midrule
POD (day-level) & 0.74 \\
POFD (day-level) & 0.33 \\
False Alarm Ratio & 0.53 \\
Frequency Bias & 1.56 \\
TSS & 0.41 \\
HSS & 0.35 \\
$F_2$ (event-level) & 0.57 \\
\bottomrule
\end{tabular}
\end{table}

\paragraph{Solar cycle robustness.}  Figure~\ref{fig:yearly} and
Table~\ref{tab:yearly} present year-by-year results from the 16-year
backtest, illustrating the algorithm's behavior across different
solar-cycle phases.  Performance is consistent across both solar cycles,
with coverage exceeding 70\% during the active phases (2011--2015,
2022--2025) and the algorithm correctly producing few alerts during the
deep minimum (2018--2019, 0--6 alerts per year with zero M-class events).

\begin{figure}[t]
\centering
\includegraphics[width=\columnwidth]{fig2_yearly_coverage.pdf}
\caption{Year-by-year M/X-class event counts (bars) and UEPI-R coverage
  (line, right axis) across Solar Cycles~24 and~25.  The 2018--2019
  solar minimum produced zero M-class events.  The 2024 coverage dip
  reflects the one-to-one matching protocol's conservatism during
  extreme activity (928 events).}
\label{fig:yearly}
\end{figure}

\begin{table}[h]
\centering
\caption{Year-by-year backtest results across 2010--2025, showing
  M/X-class event counts, alert counts, M/X coverage, false alert rate,
  and median lead time.  Years 2018--2019 had zero M-class events
  (solar minimum).}
\label{tab:yearly}
\begin{tabular}{lrrrrc}
\toprule
Year & M+X & Alerts & Cov.\% & False/d & Med.~Lead \\
\midrule
2010 &   23 &   19 &   8.7 & 0.06 & 705~min \\
2011 &  119 &  360 &  73.0 & 0.62 & 273~min \\
2012 &  136 &  358 &  76.7 & 0.55 & 176~min \\
2013 &  111 &  377 &  79.8 & 0.61 & 192~min \\
2014 &  223 &  582 &  82.6 & 0.79 & 238~min \\
2015 &  127 &  387 &  80.8 & 0.60 & 211~min \\
2016 &   16 &   86 & 100.0 & 0.16 & 307~min \\
2017 &   43 &   62 &  48.7 & 0.10 & 140~min \\
2018 &    0 &    4 &  ---  & 0.01 & --- \\
2019 &    0 &    6 &  ---  & 0.02 & --- \\
2020 &    2 &   15 &  50.0 & 0.04 & 797~min \\
2021 &   29 &   76 &  55.6 & 0.14 & 352~min \\
2022 &  196 &  432 &  78.3 & 0.61 & 252~min \\
2023 &  336 &  722 &  85.5 & 0.86 & 278~min \\
2024 &  928 &  742 &  54.6 & 0.39 & 215~min \\
2025 &  349 &  555 &  74.9 & 0.59 & 234~min \\
\bottomrule
\end{tabular}
\end{table}

Two years stand out as anomalies.  The low coverage in 2010 (8.7\%)
reflects the combination of limited legacy GOES data coverage
($\sim$50\% valid samples due to quality-flag masking) and the 48-hour
warm-up requirement, which together leave few evaluable windows during
a year with only 23 events.
The notably lower coverage in 2024 (54.6\%) reflects the extreme activity
level of that year: with 928 M/X events (the highest annual count in the
GOES era), the one-to-one matching protocol becomes particularly
restrictive during sustained storm periods, where a single long-duration
alert may span multiple flares but can only claim credit for one.
X-class coverage in 2024 remained 96.3\% (52/54), confirming that the
algorithm's detection of the most intense events is robust even during
extreme activity.

\subsection{Lead Time Distribution}
\label{sec:leadtime}

Figure~\ref{fig:leadtime} and Table~\ref{tab:leadtime} present the
cumulative lead time distribution for the 1{,}364 matched flares under
overlap matching.  The distribution is concentrated at short lead times:
80\% of detections occur within 4~hours of flare onset, and 92\% within
6~hours.  The practical early-warning window is 1--6~hours for the vast
majority of events.

\begin{figure}[t]
\centering
\includegraphics[width=\columnwidth]{fig1_lead_time_cdf.pdf}
\caption{Cumulative lead time distribution for the precision-optimized
  configuration (overlap matching, 1{,}364 matched flares).  Dashed
  lines indicate the 50th, 80th, and 90th percentiles.  The steep
  rise between 1--4~hours reflects the characteristic pre-flare regime
  duration.}
\label{fig:leadtime}
\end{figure}

\begin{table}[h]
\centering
\caption{Cumulative lead time distribution for the precision-optimized
  configuration (overlap matching, 1{,}364 matched flares).  Five flares
  (0.4\%) have lead times exceeding 24~hours, extending to a maximum of
  35~hours.}
\label{tab:leadtime}
\begin{tabular}{rcc}
\toprule
Lead time & Cumulative \% & Incremental \% \\
\midrule
$\leq$1~h  & 19.5 & 19.5 \\
$\leq$2~h  & 39.7 & 20.2 \\
$\leq$3~h  & 62.5 & 22.8 \\
$\leq$4~h  & 80.4 & 18.0 \\
$\leq$6~h  & 92.3 &  11.9 \\
$\leq$8~h  & 96.0 &  3.7 \\
$\leq$12~h & 98.2 &  2.2 \\
$\leq$24~h & 99.6 & 1.4 \\
\bottomrule
\end{tabular}
\end{table}

M-class and X-class flares exhibit similar median lead times (2.4~hours
and 2.3~hours, respectively), indicating that the pre-flare XRS signature
is comparably detectable across the full range of major flare intensities.

\subsection{Live Deployment}
\label{sec:live}

UEPI-R has been operating continuously since January~2026 against the
NOAA/SWPC real-time XRS feeds, running via automated workflows with a
15-minute polling interval.  All alerts are timestamped and committed to a
public Git repository before flare occurrence, providing cryptographic
proof of prediction timing via immutable commit hashes.

As of 2026 February~12, the live system has produced 6 alerts, of which
5 were verified as true positives against the NOAA official flare event
list (M-class hit rate: 83.3\%, median lead time: 4\,h\,57\,min).
The verified hits include:

\begin{itemize}
  \item M1.8 flare on 2026 Feb~8 detected with 6\,h\,34\,min lead time
  \item M1.7 flare on 2026 Feb~8 detected with 6\,min lead time
  \item M2.8 flare on 2026 Feb~9 detected with 17\,min lead time
  \item M1.4 flare on 2026 Feb~11 detected with 4\,h\,57\,min lead time
  \item M1.4 flare on 2026 Feb~12 detected with 13\,h\,29\,min lead time
\end{itemize}

While the sample size remains small, these initial results are consistent
with the backtest performance and---critically---demonstrate end-to-end
operational capability including real-time data ingestion, stateful
processing, alert generation, and automated verification against official
NOAA records.

\subsection{Diagnostic Ceiling Analysis}

To understand the theoretical limits of the approach, we performed a
diagnostic analysis of all 2{,}638 M/X flares to determine what fraction
produce any detectable pre-flare signal in the XRS data under the UEPI-R
framework.  We find that 93.3\% of flares (2{,}461/2{,}638) produce a
commitment-level signal at some point before onset.  The remaining 6.7\%
either lack sufficient precursor emission, occur during data gaps, or
arise from configurations where the XRS flux does not exhibit the
characteristic pre-flare regime shift.

The gap between the diagnostic ceiling (93.3\%) and the achieved
coverage (64--78\%) is attributable to two factors: (i)~the matching
protocol, which limits credit during storm clusters where multiple
flares follow a single alert, and (ii)~the alert commitment criteria,
which must be restrictive enough to maintain acceptable precision.

\subsection{Alert Duration and Flare Timing}
\label{sec:duration}

An analysis of 2{,}493 alerts across five active solar years (2011,
2013, 2014, 2022, 2024) reveals a statistically significant
\emph{negative} correlation between alert duration and lead time
(Spearman $r = -0.29$, $p < 10^{-20}$): longer alerts tend to have
shorter lead times.  This counterintuitive result reflects a structural
property of the detection architecture.

Table~\ref{tab:duration} shows the relationship stratified by alert
duration quartile.  The longest-duration alerts (Q4, $>$5~hours) have
a median lead time of only 1.7~hours, with 83.6\% of matched flares
occurring \emph{during} the active alert window.  In these cases, the
flare's own flux spike sustains the detector's commitment state,
extending the alert well past its minimum hold time.  Conversely, the
shortest alerts (Q1--Q2, 3.2--4.0~hours) have longer lead times
(median 4.7--5.3~hours), with only 38--40\% of flares occurring during
the alert.  These short-duration alerts represent the operationally most
valuable detections: the system identifies a pre-flare regime shift, the
alert decays as the transient instability subsides, and the flare arrives
hours later.

\begin{table}[h]
\centering
\caption{Alert duration quartile analysis (1{,}027 matched alerts
  across 2011, 2013, 2014, 2022, 2024).  Longer alerts are associated
  with shorter lead times and a higher fraction of flares occurring
  during the alert window.}
\label{tab:duration}
\begin{tabular}{lcccc}
\toprule
Quartile & Duration & Med.\ Lead & \% During \\
\midrule
Q1 (shortest) & 3.2--3.7\,h & 282\,min & 39.7\% \\
Q2            & 3.7--4.0\,h & 319\,min & 37.5\% \\
Q3            & 4.0--5.0\,h & 213\,min & 58.1\% \\
Q4 (longest)  & 5.0--35.7\,h & 103\,min & 83.6\% \\
\bottomrule
\end{tabular}
\end{table}

Hit alerts are significantly longer than false alerts (mean 4.7\,h
vs.\ 4.3\,h, Welch $t = 6.4$, $p < 10^{-9}$), indicating that
genuine pre-flare activity sustains the detector's commitment state
longer than transient fluctuations.  X-class flares show a stronger
duration--lead-time correlation (Spearman $r = -0.48$) and are more
likely to occur during the alert window (63\% vs.\ 54\% for M-class),
consistent with the more intense pre-flare signatures preceding the
largest events.

\subsection{False Alert Characterization: Sub-Threshold Activity}
\label{sec:cclass}

To assess whether ``false'' alerts (those not followed by an M1.0+
flare within 24~hours) are detecting genuine solar activity below the
M-class threshold, we cross-referenced all 1{,}466 false alerts with
the NOAA C-class flare catalog.  The results, summarized in
Table~\ref{tab:cclass}, indicate that the vast majority of nominally
false alerts are associated with real flare activity at the C-class
level.

\begin{table}[h]
\centering
\caption{C-class flare activity associated with false alerts
  (1{,}466 false alerts across 2011, 2013, 2014, 2022, 2024).
  ``During alert'' indicates a C-class flare onset within the alert
  window; ``within 24\,h'' uses the same hazard window as M/X
  matching.}
\label{tab:cclass}
\begin{tabular}{lrc}
\toprule
Category & $N$ & \% \\
\midrule
C-class during alert window  & 920  & 62.8\% \\
C-class within 24\,h         & 1{,}404 & 95.8\% \\
Truly false (no C/M/X in 24\,h) & 62 & 4.2\% \\
\bottomrule
\end{tabular}
\end{table}

Only 4.2\% of false alerts (62 out of 1{,}466) occur during periods
with no cataloged C-, M-, or X-class flare activity.  The detector is
responding to genuine solar X-ray enhancements in 95.8\% of all
cases---the activity simply does not escalate to M-class.

The median lead time from alert onset to the first C-class flare is
2.7~hours, comparable to the M/X-class lead times, with 65.5\% of
first C-class flares occurring during the active alert window.
Stronger C-class activity (C5.0--C9.9) arrives faster: median
lead time of 2.2~hours with 73\% occurring during the alert, compared
to 3.7~hours and 60\% for weaker activity (C1.0--C4.9).  Notably,
33.5\% of false alerts have a C7+ flare (within one order of
magnitude of M-class) as their strongest associated event within 24
hours.

These findings suggest that UEPI-R's effective precision is
substantially higher than the M/X-only metric of $\sim$40\% indicates.
If C-class detections are considered operationally relevant (as they
are for many applications including HF radio propagation and satellite
charging), the system's true positive rate approaches 96\%, with only
$\sim$4\% of alerts representing genuine noise triggers.

% ========================================================================
\section{Comparison with Existing Systems}
\label{sec:comparison}
% ========================================================================

Direct comparison of flare forecasting systems is complicated by
differences in forecast horizons, evaluation metrics, flare class
definitions, event catalogs, and validation periods
\citep{barnes2016, leka2019, bloomfield2012}.  Nevertheless, we attempt
a broad contextualization of UEPI-R's performance.
Table~\ref{tab:comparison} summarizes key systems spanning operational
forecasters, magnetogram-based ML models, and XRS-only approaches.

\begin{table*}[t]
\centering
\caption{Comparison of solar flare forecasting systems for $\geq$M-class
  events.  TSS values marked with $^\dagger$ are laboratory/test-set
  values; operational TSS is typically lower (see text).  ``24\,h binary''
  denotes a daily will-flare/won't-flare classification; ``continuous''
  denotes minute-resolution alerting with measured lead times.}
\label{tab:comparison}
\begin{tabular}{llcccc}
\toprule
System & Input data & TSS (M+) & Operational? & Forecast type \\
\midrule
NOAA/SWPC & Expert + magnetograms & $<$climatology & Yes & 24\,h prob. \\
DAFFS \citep{leka2018} & HMI + GONG + flare hist. & --- & Yes (CCMC) & 24\,h prob. \\
Deep Flare Net \citep{nishizuka2021} & HMI + AIA & 0.80--0.87$^\dagger$ & Yes (RWC-J) & 24\,h binary \\
SolarFlareNet \citep{abduallah2023} & HMI magnetograms & 0.84$^\dagger$ & No & 24\,h binary \\
SwinTCN-Att \citep{zhang2025} & HMI SHARP (16 params) & 0.88$^\dagger$ & No & 24\,h binary \\
LLMFlareNet \citep{li2026} & HMI SHARP (10 params) & 0.57--0.72$^\dagger$ & Yes & 24\,h binary \\
Moirai2 \citep{riggi2025} & \textbf{GOES XRS only} & \textbf{0.74}$^\dagger$ & No & 24\,h binary \\
\midrule
UEPI-R (this work) & \textbf{GOES XRS only} & \textbf{0.41} & Yes & Continuous \\
\bottomrule
\end{tabular}
\end{table*}

\subsection{NOAA/SWPC Operational Forecasts}

The NOAA/SWPC issues daily probabilistic forecasts for M- and X-class
flares.  A comprehensive 26-year verification by \citet{camporeale2025}
found that SWPC forecasts do not outperform zero-cost baselines such as
persistence and climatology, exhibit severe calibration issues, and
produce false alarm ratios exceeding 90\% for significant flare events.
By contrast, UEPI-R achieves a false alarm ratio of approximately 61\%
(precision 39.6\%) while maintaining 97.2\% X-class coverage.
Moreover, SWPC forecasts are issued once daily with a 24-hour horizon,
whereas UEPI-R provides continuous, minute-resolution alerts with lead
times ranging from minutes to $>$12~hours.  The two systems are thus
complementary: SWPC provides daily probabilistic context while UEPI-R
provides real-time onset warnings.

\subsection{Magnetogram-Based ML Systems}

The highest reported TSS values come from systems that leverage SDO/HMI
magnetogram features.  SwinTCN-Att \citep{zhang2025} combines a Swin
Transformer with temporal convolutional networks over 16 SHARP magnetic
field parameters, achieving TSS~$\approx 0.88$ for M+ prediction.
SolarFlareNet \citep{abduallah2023} achieves TSS~$\approx 0.84$ using
HMI data with a deep neural network.  Deep Flare Net
\citep{nishizuka2021}, operational at the Regional Warning Center Japan,
reports TSS~$= 0.80$--$0.87$ on test sets.  \citet{bobra2015} report
TSS~$\approx 0.76$ using HMI vector fields with support vector machines.
LLMFlareNet \citep{li2026}, a BERT-based system operational since 2022,
achieves TSS~$= 0.72$ under cross-validation but TSS~$= 0.57$ in
daily operational mode---illustrating the lab-to-operations gap.  The
Discriminant Analysis Flare Forecasting System (DAFFS;
\citealt{leka2018}), operational at NASA/CCMC, uses a non-parametric
statistical approach on HMI and GONG data.

These systems achieve higher point-prediction TSS than UEPI-R for the
binary 24-hour classification task.  However, three caveats apply.
First, laboratory TSS values are often evaluated on balanced or carefully
partitioned test sets that may not reflect operational conditions;
\citet{kubo2017} document that Deep Flare Net's operational TSS degrades
to $\sim$0.24.  Second, all magnetogram systems depend on SDO/HMI data
(Section~\ref{sec:sdo}).  Third, none provides continuous onset timing or
measured lead times---they classify whether a flare will occur in the next
24~hours, not \emph{when}.

\subsection{XRS-Only Approaches}

\citet{riggi2025} provide the most directly comparable benchmark,
applying the Moirai2 foundation time-series model to GOES XRS data
(two-channel, 1440-point windows at one-minute cadence) for 24-hour
M+ binary classification.  Their XRS-only model achieves
TSS~$= 0.74 \pm 0.01$ and HSS~$= 0.69 \pm 0.02$, substantially
outperforming their own magnetogram-based image (TSS~$= 0.65$) and
video (TSS~$= 0.60$) models.  This result independently validates
UEPI-R's foundational premise: the GOES XRS flux stream carries
sufficient predictive information for M/X-class forecasting, without
recourse to spatial solar data.

UEPI-R's day-level TSS of $\approx$0.41 is lower than Moirai2's
0.74, which is expected given the architectural differences: Moirai2
is a 311-million-parameter transformer trained on 9~billion time-series
observations, optimized end-to-end for the 24-hour binary task.  UEPI-R
uses a small, fixed set of physics-motivated diagnostics with $\sim$10
tunable thresholds.  However, UEPI-R provides capabilities that Moirai2
does not: continuous real-time alerting with measured lead times
(median 2.4--4.3~h), multi-tier alert states, and operational deployment
requiring only commodity hardware and no GPU inference.

\subsection{UEPI-R's Niche}

UEPI-R occupies a distinct niche in the flare forecasting landscape:

\begin{enumerate}
  \item \textbf{Real-time onset warning} (minutes to hours) rather than
    daily probability forecasts.  No other operational system provides
    continuous alert timing with measured lead times.
  \item \textbf{Minimal data requirements} (GOES XRS only), ensuring
    resilience against magnetogram pipeline failures and SDO end-of-life
    (Section~\ref{sec:sdo}).
  \item \textbf{Cross-cycle stability.}  The physics-based architecture
    avoids distribution shift between solar cycles---a known failure mode
    for ML systems trained on Cycle~24 and deployed during Cycle~25.
  \item \textbf{Near-perfect X-class detection} (97--99\%), making it
    particularly valuable for the highest-impact events.
\end{enumerate}

The 97.2\% X-class coverage is, to our knowledge, the highest reported
for any system validated over a multi-cycle period.  This likely reflects
the fact that X-class flares are preceded by the most pronounced
pre-flare coronal signatures, which are reliably detectable in
disk-integrated XRS data.

% ========================================================================
\section{Discussion}
\label{sec:discussion}
% ========================================================================

\subsection{Limitations}

Several limitations of the current system should be noted:

\paragraph{Disk-integrated data.}  UEPI-R uses full-disk irradiance and
cannot distinguish between flare-productive and quiescent active regions.
During multi-region activity, the system may produce alerts driven by one
region while a flare occurs in another, and the one-to-one matching
protocol may under-count true detections in these scenarios.

\paragraph{Storm-period saturation.}  During sustained high-activity
periods (e.g., 2024 with 928 M/X events), the alert system tends toward
quasi-continuous elevated states.  While individual flares are still
detected, the one-to-one matching metric penalizes this behavior, and
the operational utility of an ``always on'' alert is reduced.

\paragraph{Short lead times.}  While the median lead time is
2.4--4.3~hours (depending on matching method), the 10th~percentile is
approximately 27~minutes under overlap matching.  For approximately 20\%
of detected flares, the warning arrives less than one hour before onset.
These short-lead detections typically correspond to impulsive flares with
minimal precursor phase.

\paragraph{Limited live validation.}  As of this writing, the live
deployment spans only three weeks with 6 alerts.  Longer operational
history is needed to establish live performance statistics with confidence.

\subsection{SDO Dependency and Instrument Continuity}
\label{sec:sdo}

A strategic consideration for flare forecasting systems is instrument
continuity.  The vast majority of modern forecasting systems---including
Deep Flare Net, SolarFlareNet, DAFFS, LLMFlareNet, and SwinTCN-Att---depend
on photospheric magnetogram data from SDO/HMI, which has been the sole
source of high-cadence, full-disk vector magnetograms since its launch
in 2010.  SDO is currently operating beyond its design lifetime, and
no replacement mission with equivalent magnetogram capability has been
confirmed beyond $\sim$2030.  When SDO ceases operation, every
magnetogram-dependent forecasting system will lose its primary input.

GOES XRS data, by contrast, are provided by an operational satellite
constellation with guaranteed continuity through the GOES-U mission
(planned through the 2030s) and successors.  Multiple GOES spacecraft
operate simultaneously, providing built-in redundancy.  UEPI-R's
exclusive reliance on XRS data thus provides a forecasting capability
that is robust against the single largest instrument-continuity risk
facing the space weather community.  This makes XRS-only approaches
particularly valuable as either standalone systems or as fallback
components in ensemble architectures.

\subsection{Comparison with the ``No-Skill'' Baseline}

A critical question for any flare forecasting system is whether it
provides skill above simple persistence or climatological baselines
\citep{bloomfield2012}.  UEPI-R's false alert rate of $\sim$0.37~day$^{-1}$
combined with 64--71\% coverage is substantially better than a naive
threshold detector applied to the raw XRS flux (which would either
produce far more false alerts at comparable coverage, or far lower
coverage at comparable false rates).  The precision-optimized
configuration achieves an $F_2$ score of 0.569 under 1:1 matching,
confirming meaningful skill.

Moreover, the C-class analysis of Section~\ref{sec:cclass} demonstrates
that the system's effective false-positive rate is far lower than the
M/X-only metric suggests: only 4.2\% of alerts occur during periods
with no cataloged flare activity at any class.  The remaining 95.8\%
of ``false'' alerts are detecting real C-class activity---the solar
corona is genuinely in an elevated state, but the energy release does
not reach M-class.  This distinction is operationally important:
C-class flares affect HF radio propagation and contribute to cumulative
radiation exposure, making many of these alerts actionable even when
M-class thresholds are not reached.

\subsection{Future Directions}

Several avenues for improvement are under investigation:

\begin{enumerate}
  \item \textbf{Multi-flare matching.}  Allowing alerts to claim credit
    for multiple flares during storm periods would better reflect
    operational utility but requires careful metric design to avoid
    inflation.
  \item \textbf{Probabilistic output.}  Converting the binary alert
    into a continuous probability would enable direct comparison with
    SWPC forecasts and facilitate downstream decision-making.
  \item \textbf{EUV augmentation.}  Incorporating GOES/SUVI or SDO/AIA
    EUV data could improve specificity during multi-region activity,
    while preserving the system's operational robustness by treating
    spatial data as optional.
  \item \textbf{Extended live validation.}  Continued operation through
    Solar Cycle~25 maximum will provide statistically robust live
    performance estimates.
\end{enumerate}

% ========================================================================
\section{Conclusion}
\label{sec:conclusion}
% ========================================================================

We have presented UEPI-R, a real-time early warning system for M- and
X-class solar flares that operates exclusively on GOES XRS irradiance
data.  Using causal regime detection with physics-motivated diagnostics
and no spatial solar data, the system achieves 64--71\% M/X coverage
and 97.2\% X-class coverage over a 16-year validation period
(2010--2025), with a day-level TSS of 0.41, a median lead time of
2.4~hours (80\% of detections providing $\geq$1~hour warning),
precision of $\sim$39\%, and a false alert rate of
$\sim$0.37~day$^{-1}$.

These results demonstrate that the GOES XRS flux stream contains
substantial predictive information for major solar flares---information
that is largely unexploited by existing operational systems.  Analysis
of nominally false alerts reveals that 95.8\% are associated with
C-class flare activity, with only 4.2\% representing true noise
triggers---the system detects genuine solar instability in nearly all
cases.  UEPI-R's minimal data requirements, strict causality, and
real-time capability make it well-suited as a complement to existing
magnetogram-based forecasting systems, providing continuous onset
warnings where current systems provide only daily probabilities.

The near-perfect X-class detection rate (97.2\%) is particularly
noteworthy.  For the highest-impact space weather events---those most
likely to affect satellite operations, aviation, and power grids---UEPI-R
provides reliable advance warning using only the most widely available
and redundant solar monitoring data.

\subsection*{Data Availability}

The GOES XRS data used in this study are publicly available from NOAA's
National Centers for Environmental Information
(\url{https://www.ncei.noaa.gov/data/goes-space-environment-monitor/})
and the SWPC real-time archives
(\url{https://services.swpc.noaa.gov/json/goes/primary/}).
The NOAA flare event lists are available from
\url{https://www.ngdc.noaa.gov/stp/space-weather/solar-data/solar-features/solar-flares/x-rays/goes/xrs/}.
The UEPI-R algorithm is proprietary (U.S.\ Provisional Patent Application
No.\ 63/949,419); live alerts and verification logs are
published at \url{https://github.com/quantexenergy/UEPI-R-solar-feed}.

\begin{acknowledgments}
The author acknowledges the NOAA Space Weather Prediction Center and the
NOAA National Centers for Environmental Information for maintaining and
distributing the GOES XRS data products that make this work possible.
\end{acknowledgments}

\bibliography{references}

\end{document}
